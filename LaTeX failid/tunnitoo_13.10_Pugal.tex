\documentclass[a4paper]{article}
\usepackage[estonian]{babel}
\usepackage[utf8]{inputenc}
\usepackage[T1]{fontenc}
\usepackage{mathtools}
\usepackage{amssymb,amsmath}
\usepackage{siunitx}

%tanelp@ioc.ee

\title{Tuletis}
\author{Siim Erik Pugal}
\date{13. oktoober 2017. a}


\begin{document}
\maketitle

\section*{Sissejuhatus}
 Funktsiooni \textbf{tuletis} on \textbf{matemaatilise analüüsi} üks põhimõisteid. Funktsiooni tuletis mingil kohal näitab selle funktsiooni väärtuse muutumise kiirust funktsiooni \textbf{argumendi} muutumisel — täpsemalt, funktsiooni tuletis on funktsiooni väärtuse muudu ja argumendi muudu suhte \textbf{piirväärtus} argumendi muudu lähenemisel nullile.

 \textbf{Füüsikas} on \textbf{nihke} tuletiseks aja järgi \textbf{hetkkiirus}, kiiruse tuletiseks omakorda \textbf{kiirendus}.
 
 Reaalarvulise argumendiga ning reaalarvuliste väärtustega funktsiooni korral on selle funktsiooni tuletiseks mingil kohal selle \textbf{funktsiooni graafiku puutuja tõus} sellel kohal.
 
 Matemaatilise analüüsi eeskujul on tuletise mõistet mitmel viisil üldistatud teistesse matemaatika valdkondadesse. Käesolev artikkel käsitleb põhiliselt reaal- või kompleksmuutuja funktsiooni tuletist matemaatilise analüüsi tähenduses.
 
\section{Määratlus}
\subsection{Tuletis antud kohal}
Olgu antud reaalarvuliste väärtustega funktsioon \textbf{\textit{f}}  ning \textbf{\textit{x}} mõni reaalarv funktsiooni määramispiirkonnast. Kui leidub (lõplik või lõpmatu) piirväärtus 
\begin{equation}
\label{N2S}
\mathit{\lim_{x\to\infty} \frac{f(x+h)-f(x)}{h}},
\end{equation} 
siis seda nimetatakse funktsiooni \textbf{\textit{f}} tuletiseks kohal \textbf{\textit{x}} ning tähistatakse sümboliga \textbf{\textit{f'(x)}}.

 Tavaliselt määratletakse funktsiooni tuletis vaid tema määramispiirkonna sisepunktides, s. t. eeltoodud definitsiooni lisatakse veel eeldus, et \textbf{\textit{x}} on hulga \textbf{\textit{D}} sisepunkt.

 Kui funktsioonil \textbf{\textit{f}} on lõplik tuletis kohal \textbf{\textit{x}}, nimetatakse funktsiooni \textbf{\textit{f}} diferentseeruvaks kohal \textbf{\textit{x}}.
 
 Samamoodi defineeritakse tuletis ja diferentseeruvus ka kompleksmuutuja funktsiooni korral, s. t. juhul $f : D \rightarrow \mathbb{C}$, kus $D \subset \mathbb{C}$.



\end{document}